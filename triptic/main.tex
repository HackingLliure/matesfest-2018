\documentclass{beamer}
\mode<presentation>
{
  \usetheme{default}        % or try Darmstadt, Madrid, Warsaw, ...
  \usecolortheme{beaver}    % or try albatross, beaver, crane, ...
  \usefonttheme{serif}      % or try serif, structurebold, ...
  \setbeamertemplate{navigation symbols}{}
  \setbeamertemplate{caption}[numbered]
} 

\usepackage[english]{babel}
\usepackage[utf8x]{inputenc}


\title[Crypto]{}
\author{Hacking LLiure Squad}
\institute{UB}
\date{Matefest 2018}

\begin{document}

% Cara 1
\begin{frame}
  \begin{columns}[t]
  
  \begin{column}{.05\linewidth}
  \end{column}

  \begin{column}{.30\linewidth}
    \begin{block}{Bloque 1}
      In LaTeX, we don’t need to do such copy and paste thing. LaTeX has
      different packages which automatically generates dummy text in our
      document. You can generate them with just a few lines of code.
    \end{block}
  \end{column}
  
  \begin{column}{.30\linewidth}
  \begin{block}{Bloque 2}
  Bloque en la segunda columna
  \end{block}
  \end{column}
  
  \begin{column}{.30\linewidth}
  \begin{block}{Bloque 2}
  Bloque en la segunda columna
  \end{block}
  \end{column}
  
  \begin{column}{.05\linewidth}
  \end{column}

  \end{columns}
\end{frame}

% Cara 2
\begin{frame}
  \begin{columns}[t]
  
  \begin{column}{.05\linewidth}
  \end{column}

  \begin{column}{.30\linewidth}
  \begin{block}{Bloque 1}
  Bloque de la primera columna
  \end{block}
  \end{column}
  
  \begin{column}{.30\linewidth}
  \begin{block}{Bloque 2}
  Bloque en la segunda columna
  \end{block}
  \end{column}
  
  \begin{column}{.30\linewidth}
  \begin{block}{Bloque 2}
  Bloque en la segunda columna
  \end{block}
  \end{column}
  
  \begin{column}{.05\linewidth}
  \end{column}

  \end{columns}
\end{frame}

\end{document}
